\documentclass[12pt]{article}
%\documentclass[titlepage,12pt]{article}
% definimos tamaño de fuente 12pt
% PAQUETERÍAS 
\usepackage[letterpaper,top=2cm,bottom=2cm,left=3cm,right=3cm,marginparwidth=1.75cm]{geometry} %modifica el tamaño de márgenes
\usepackage{amssymb,latexsym,amsmath,amsthm,amsfonts} %matemáticas
\usepackage{multicol} %para tablas
\usepackage{multirow} % para tablas
\usepackage[utf8]{inputenc} % acentos, tildes, etc
\RequirePackage[spanish]{babel} % identifica que escribimos en español
\usepackage{graphicx} % utilizar acciones de imágenes
\usepackage{fancyhdr} % estilo
%\usepackage{xcolor} % colores
\usepackage[table,xcdraw]{xcolor} %colores para tablas
\usepackage{cellspace} %tablas
\usepackage{caption,subcaption} % permite usar subfigures
\usepackage{listings} % colocar código
\usepackage{hyperref} % enlaces de internet
\usepackage{booktabs}
\usepackage{parskip} %espacio entre líneas
% información del equipo y número de tarea
\title{\textbf{TAREA número 1. Equipo 1}}
\author{Integrante 1\\ Integrante 2\\ Integrante 3}
\date{}

% CUERPO DEL DOCUMENTO (DONDE ESCRIBIMOS TODO Y AÑADIMOS IMÁGENES, TABLAS, ETC)
\begin{document}
\maketitle % imprime la información

% secciones sin numerar ya que tiene un asterisco
\section*{\\Problema 1}
Este es el primer problema que tenemos que resolver, damos un ejemplo de la tabla de datos:
%\subsection*{Ejercicio 1.1}


\begin{table}[h]
 % !: indica a latex que tome en cuenta los parámetros que le siguen
 % h: indicamos que la tabla aparezca aquí (here), justo en esta parte del documento.
 % b: (bottom) tratará de colocar la tabla en la parte inferior de la página
 % t: (top) " " parte superior de la página.
        \centering % alineamos hacia el centro
        %\flushleft % alineamos hacia la izquierda
        %\flushright % alineamos hacia la derecha
        \small % minimizamos el tamaño de letra
        \begin{tabular}{|c|c|c|c|c|c|c|} 
        % # columnas y el símbolo | delimita con líneas verticales...(3)*
        % || doble línea vertical
        \hline
          Alimento &Ración&Energía&Proteínas& Calcio&Precio& Raciones\\
          & & (Kcal)& (g)& (mg)& (\$)& máximas\\
          \hline\hline %doble línea
            Avena & 28g& 110& 4& 2& 3& 4\\
            \hline %una línea
            Pollo & 100g& 205& 32& 12& 24& 3\\
            \hline
            Huevos& 2pz& 160& 13& 54& 13& 2\\
            \hline
            Leche& 250ml& 150& 8& 285& 9& 8\\
            \hline
            Pastel& 170g& 420& 4& 22& 20& 2\\
            \hline
            Frijoles& 260g& 260& 14& 80& 19& 2\\
            \hline
        \end{tabular}
        \caption{Otro problema de dieta} % descripción de tabla y numera
        %\label{nombre-para-identificar} 
    \end{table}

Ahora vamos a escribir como se plantearía un modelo, con su función objetivo, las restricciones, la condición de no negatividad de las variables.\\ %salto de línea
Entonces planteamos lo siguiente:

\begin{equation*}
\begin{array}{lll}
    \mbox{Minimizar}
    & \sum_{i=1}^{n}f(\mbox{\boldmath{$x_i$}}) &\\
    \mbox{sujeto a: }
    & g_i(\mbox{\boldmath{$x$}}) \leq 0; &
    i=1,\ldots,m\\
    & g_i(\mbox{\boldmath{$x$}}) \geq 0; &
    i=1,\ldots,m\\
    & h_k(\mbox{\boldmath{$x$}}) = 0; &
    k=1,\ldots,p\\
    & \forall x_j \geq 0; & j=1,\ldots,n
    \end{array}
\end{equation*}

Colocamos una imagen representando el espacio de soluciones, elaborada en Geogebra:

\begin{figure}[ht] %estamos utilizando here y top ya que se posiciono arriba, pero podemos sólo utilizar h 
\centering
\includegraphics[width=.85\textwidth]{Ejercicio 2.2.png}
\caption{Cuarto punto}
\label{cuarta}
\end{figure}


{\color{red}Solución: }\\
La solución del problema puede escribir así.

O si ocupamos dos imágenes

\begin{figure}[ht] %englobamos las imágenes
\centering
    \begin{subfigure}[h]{0.45\textwidth} % primera imagen 
        \centering
        \includegraphics[width=0.85\textwidth]{Ejercicio 2.2.png}
        \caption {Está es la imagen 1}
        \label{fig: imagen 1}
    \end{subfigure}
    \hspace{0.1cm} % espaciado entre imágenes
    \begin{subfigure}[h]{0.45\textwidth} % segunda imagen 
        \centering
        \includegraphics[width=0.65\textwidth]{frog.jpg}
        \caption {Está es la imagen 2}
        \label{fig: imagen 2}
    \end{subfigure}
 \caption{Estamos mostrando el ejemplo de dos imágenes en un arreglo.}
 \label{dos imágenes}
\end{figure}

\newpage

\section*{\\Expresiones matemáticas}
%--- Algunas operaciones
$$ \sum_{i=1}^{N} x_i$$
$$ \prod_{i=1}^{n}$$
\[\int_a^b x^2 dx\] % si necesitamos más de una integral agrega otra i -> iint
\[\dfrac{n^2}{n^2(n+z)^2}\]

$\frac{1}{k} \log_{2} c(f)$ es la medida \dots
\[\frac{1}{2}\frac{1+x}{1-x}+\frac{x^2}{(x-1)^3}+\frac{(x^2+1)^{\frac{1}{3}}}{y_2}-\frac{x-y}{1+\frac{1}{y}}\]
\[(a+b)^{n}=\binom{n}{0} a^{n}+\tbinom{n}{1}a^{n-1}b+ \dots + \binom{n}{n}n^n\]

\[ \cfrac{1}{\sqrt{2}+ \cfrac[l]{2}{\sqrt{2}+ \cfrac[r]{3}{\sqrt{2}+ \dots }}} \]
% los valores l o r sirven para que los numeradores aparezcan justificados respectivamente
\[\sum_{i=1}^\infty \frac{1}{n^2}=\frac{1}{n^2}=\frac{pi^2}{6}\]
$$\sqrt[3]{2x}$$
\[\lim_{x \to 0}\] %una forma 
\[\underset{x\to 0}{\lim}\] % segunda forma de escribir el límite
\par 
% ------ TAMAÑOS DE SIMBOLOS
Podemos también notar que a veces el tamaño de los paréntesis o llaves no son los adecuados para nuestras operaciones, por lo que podemos optar por escribir lo siguiente:
\[(\frac{1}{2})\]
% el valor l significa left y r right, puede ser big o Big depende el tamaño
\[\Bigl(\frac{1}{2} \Bigr)\]
\[\Bigl\{\frac{1}{2} \Bigr\}\]
\[\Bigl[\frac{1}{2} \Bigr]\]
\par
%------ ALINEACIÓN EN OPERACIONES
Ahora para alineación de múltiples ecuaciones podemos usar los siguientes ambientes:\\
Opción 1:
% \nonumber elimina la numeración automática de la ecuación 
\begin{gather}
    2x_{1}+3x_{2}\leq 12\\
    2x_{1} - x_{2} \leq 9 \nonumber \\
    x_{1},x_{2} \geq 0
\end{gather}

Opción 2: está opción alineará respecto al símbolo que este entre \& simbol \&
\begin{eqnarray*}
    2x_{1}+3x_{2} &\leq& 12 \\
    2x_{1} - x_{2} &=& 9 \\
    x_{1},x_{2} &\geq& 0
\end{eqnarray*}
Opción 3: con el comando \textbf{aling} y el uso de \&, si queremos que no estén numeradas usamos \textbf{align*}:
\begin{align*}
    2x_{1}+3x_{2} \leq 12\\
    2x_{1} - x_{2} = 9 \\
    x_{1},x_{2} \geq 0
\end{align*}
\par
Ahora vamos a ver la creación de matrices.\\
Opción 1:
\begin{equation}
    \begin{matrix}
     3 &  2 \\
     4 & 5 \\
    \end{matrix}
\end{equation}
Opción 2: el uso de paréntesis, estos se adaptan automáticamente
\begin{equation*}
    \begin{pmatrix}
     x_{11} & x_{12} \\
     x_{21} & x_{22} 
    \end{pmatrix}
\end{equation*}

\begin{equation}
% definimos la primera matriz
    \begin{pmatrix}
     x_{11} & x_{12} \\
     x_{21} & x_{22} 
    \end{pmatrix}
    %definimos la segunda
     \begin{pmatrix}
     x_{11} & x_{12} \\
     x_{21} & x_{22} 
    \end{pmatrix}
\end{equation}

Opción 3: el uso de corchetes.
\begin{equation}
   \begin{bmatrix}
    x_{11} & x_{12} & x_{13} \\
     x_{21} & x_{22} & x_{23}
    \end{bmatrix}
\end{equation}
Opción 4:
\begin{equation}
   \begin{vmatrix}
     x_{11} & x_{12} & x_{13} \\
     x_{21} & x_{22} & x_{23}\\
     x_{31} & x_{32} & x_{33}
    \end{vmatrix}
\end{equation}

Opción 5:
\begin{equation}
   \begin{Vmatrix}
     x_{11} & x_{12} & x_{13} \\
     x_{21} & x_{22} & x_{23}\\
     x_{31} & x_{32} & x_{33}
    \end{Vmatrix}
\end{equation}

\subsection{Letras del alfabeto griego}
\[ \alpha, A, \beta, B, \gamma, \Gamma, \delta, \Delta, \theta, \Theta, \dots
\]


\section{Listas y numeración}
\subsection{Listas no numeradas: Entorno itemize}
Veamos un ejemplo de una lista de objetos, esta es la manera más sencilla y recurrente para generar una lista no numerada.
\begin{itemize}
    \item Este es el primer objeto
    \item Observemos que la indentación es generada automáticamente sin importar el tamaño del texto.
    \item Recuerda usar la instrucción item para numerar cada elemento de la lista.
\end{itemize}
Ahora vamos a probar viendo una lista anidada:
\begin{itemize}
    \item 1ra entrada del primer nivel
    \item 2da entrada del primer nivel
    \begin{itemize}
        \item 1ra entrada del segundo nivel
        \item 2da entrada del segundo nivel
        \begin{itemize}
            \item 1ra entrada del tercer nivel
            \item 2da entrada del tercer nivel
            \begin{itemize}
                \item 1ra entrada del cuarto nivel
                \item 2da entrada del cuarto nivel
            \end{itemize}
        \end{itemize}
    \end{itemize}
\end{itemize}

Observemos que los iconos se generan por defecto, esto por la paquetería de \textbf{babel}. Pero veamos un ejemplo si queremos cambiar el tipo de viñetas.

{\renewcommand{\labelitemi}{$\triangleright$}
\begin{itemize}
\item Item 1
\item Item 2
\item Item 3
\end{itemize}}

\subsection{Listas numeradas: El entorno enumerate}
Veamos el entorno \textbf{enumerate} con una lista simple y con una lista anidada y ver el cambio de numeración que se emplea en cada nivel.
\begin{enumerate}
    \item Este es el primer objeto
    \item Observemos que la indentación es generada automáticamente sin importar el tamaño del texto.
    \item Recuerda usar la instrucción item para numerar cada elemento de la lista.
\end{enumerate}

La lista anidada tendría una salida:
\begin{enumerate}
    \item 1ra entrada del primer nivel
    \item 2da entrada del primer nivel
    \begin{enumerate}
        \item 1ra entrada del segundo nivel
        \item 2da entrada del segundo nivel
        \begin{enumerate}
            \item 1ra entrada del tercer nivel
            \item 2da entrada del tercer nivel
        \end{enumerate}
    \end{enumerate}
\end{enumerate}

\section{Definiciones de Teoremas}
Primero vemos como escribir teoremas:
\newtheorem{thm}{Teorema}
\begin{thm}
Este es un ejemplo del entorno
para construir teoremas.
\end{thm}
\begin{thm}
Este es un segundo teorema.
\end{thm}

Después vemos como escribir corolarios:
\newtheorem{col}{Corolario}[thm]
\begin{col}[Aranda, E. 2004]
Este es un ejemplo del entorno
para construir corolarios.
\end{col}

Y por ultimo como escribir unas notas:
\newtheorem{rmk}{Nota}[section]
\begin{rmk}
Este es un ejemplo del entorno
para construir notas.
\end{rmk}

\section*{Colocar código}

\begin{lstlisting}
import numpy as np
    
def incmatrix(genl1,genl2):
    m = len(genl1)
    n = len(genl2)
    M = None #to become the incidence matrix
    VT = np.zeros((n*m,1), int)  #dummy variable
    
    #compute the bitwise xor matrix
    M1 = bitxormatrix(genl1)
    M2 = np.triu(bitxormatrix(genl2),1) 

    for i in range(m-1):
        for j in range(i+1, m):
            [r,c] = np.where(M2 == M1[i,j])
            for k in range(len(r)):
                VT[(i)*n + r[k]] = 1;
                VT[(i)*n + c[k]] = 1;
                VT[(j)*n + r[k]] = 1;
                VT[(j)*n + c[k]] = 1;
                
                if M is None:
                    M = np.copy(VT)
                else:
                    M = np.concatenate((M, VT), 1)
                
                VT = np.zeros((n*m,1), int)
    
    return M
\end{lstlisting}

Para más ejemplos \url{https://es.overleaf.com/learn/latex/Code_listing}

\newpage
Un pequeño ejemplo de una tabla generada en la página que está referenciada en \ref{tab:mi_tabla}.

\begin{table}[ht]
\centering
\begin{tabular}{ccccc}
& \multicolumn{3}{c}{\textbf{Minutos por unidad}} &  \\ \hline
\rowcolor[HTML]{EFEFEF} 
\textbf{Producto} & \textbf{Proceso 1} & \textbf{Proceso 2} & \textbf{Proceso 3} & \textbf{Utilidad diaria} \\ \hline
1 & 10 & 6 & 8 & \$ 2 \\
2 & 5 & 20 & 10 & \$ 3 \\ \hline
\end{tabular}
\caption{Prueba de tabla creada en \url{https://www.tablesgenerator.com}}
\label{tab:mi_tabla}
\end{table}

% Please add the following required packages to your document preamble:
% \usepackage{booktabs}
% \usepackage[table,xcdraw]{xcolor}
% If you use beamer only pass "xcolor=table" option, i.e. \documentclass[xcolor=table]{beamer}
\begin{table}[ht]
\centering
\caption{}
\label{tab:my-table}
\begin{tabular}{@{}ccc@{}}
\toprule
\rowcolor[HTML]{EFEFEF} 
{\color[HTML]{CB0000} \textbf{Tipo}} & {\color[HTML]{CB0000} \textbf{Producto 1}} & {\color[HTML]{CB0000} \textbf{Producto 2}} \\ \midrule
1 & 58 unidades & 23 unidades \\
2 & 67 unidades & 23 unidades \\ \bottomrule
\end{tabular}
\end{table}

\end{document}
